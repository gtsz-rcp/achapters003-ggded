\hypertarget{uxd0dcuxc591uxc758-uxb300uxc9c0}{%

\section{태양의 대지}\label{uxd0dcuxc591uxc758-uxb300uxc9c0}}



살찐 태양이 대지에 눌러 앉았던 어느 여름이었다 발치에 치이는 계절을

걷어차며 무릎까지 쌓인 계절에 묻힌 발을 꺼내는 것이 매일의 일이었다.

옆집의 미친 남자는 이때다 싶어 20대 여성을 죽여 그 아래 묻어두었다. 나의

친구는 내게 안부를 묻곤 한다. 우린 언제쯤 만날 수 있을까. 반도를 타고

낮게 흐르는 이 빛은 사라질까. 어떤 이들은 그가 가장 사랑하는 것을

박살내고 어떤 이들은 그가 가장 미워하는 것을 박살낸다. 그가 지난 해 가장

사랑했던 것은 엑스박스였다.



모두가 이 나라를 떠나고 싶어했지만 계절은 한 세기 동안 정체되었고 위대한

상상은 해안가에 머물렀다 우리가 상상할 수 있는 건 이곳이 아닌 여기까지

였다



그곳은 어떤 계절이니? 여긴 열 세번째 계절이야. 그곳은 어떤 계절이니?

이곳은 낮은 두 번째 계절이지. 한 세기 전을 기억하는 유일한 인류가 이번

세기를 버티고 있다 빌딩 옥상까지 닿은 이 땅의 계절은 유리창을 두들이며

죽은 여자의 음성을 전했다 이 계절의 쓸모가 드러나는 유일한 순간이었다

유리창이 박살나 정수리에 피를 흘리는 여자를 두고 땀을 질질 흘리던 우리는

너 때문이라 탓했다 이 행성의 껍데기가 떨리며 우리 음성을 전 세계에

전했다 우주는 우리와 한편이었어 적어도 이 행성만큼은 그 여자는 죽고

나서야 제 편을 가질 수 있었다



여자들은 모두 계절의 가장 낮은 곳에서 마주쳐 우리가 되었고, 계절에 밀려

행성 밖으로 밀려났다 우리가 넘어서지 못한 해안선의 바깥으로



높아진 우리는 늘어진 시간 위에서 더 많은 불면의 새벽을 보낼 것이다

창가의 여자들은 모조리 대지 위에 서겠지 결국 까지 살아남는 건, 이번 해를

살아서 버티는 건 나는 결코 아닐 것이다 너 또한 아닐 것이다



이번 해도 겨울은 이 행성을 지나쳐 가버렸다 높은 내 정수리엔 찰나의

서리가 녹아 흐른다 모두들 행운을 빈다

