\hypertarget{uxc885uxcc29uxc9c0}{%

\section{종착지}\label{uxc885uxcc29uxc9c0}}



경찰이 올려 보낸 수천 대의 드론이 미란다 원칙을 고지한다.



길게 늘어선 유모차의 행렬을 전경이 가로막았다. 깃발을 들고 따라 걷던

남성 몇 몇이 전경을 향해 소리질렀다. 유모차를 든 여성들은 일제히 허리를

숙여 아이들의 귀를 막았고, 유모차 가까이 딸랑이를 흔들어주었다. 한

남성이 소리쳤다. 왜 우리를 가로 막는가. 전경은 말이 없었다.



전경은 십 수명의 남성들로 구성된 군집생명체처럼 덩어리져 이 길에서 저

길로 일사 분란 하게 이동했다. 이들은 하나로 따로 떨어져 나와야만 비로소

내면의 인간성이 발동되는 듯 했지만 결국 이들을 대신해 입을 여는 건

이들의 어머니들이었다. 왜 우리 아들에게 손을 대느냐.



유모차를 미는 여성들이 아이들의 귀를 막고 소리쳤다. 왜 우릴 가로막는가.

전경들은 말이 없었다. 유모차의 줄이 전경의 대열 코 앞까지 밀고 들어왔고,

전경들의 방패가 잔뜩 긴장되었다.



전경들의 대열 뒤에서 거대한 마이크가 올라와 군중을 향해 침을 튀기며

말한다. 천이백이십일차 해산 통보입니다. 천이백이십일차 해산 통보.

여러분들은 시민들이 이용하는 찻길을 가로막고 있습니다.



뒤쪽에서 검은 팔뚝이 튀어나와 유모차를 휘어잡았다. 유모차를 잡은

여성들은 신경질적으로 주저앉아 소리를 지른다. 아이들이 비명을 지르며

경기한다. 입술이 파래진 아이가 바닥에 내동댕이쳐진다. 텅 빈 유모차들이

바닥에 나뒹군다. 한 무리를 이루었던 여성들은 모두 유모차를 집어 던지고

이제 한 아이의 어머니가 되어 전경의 군집을 벗어나기 위해 안간힘을 쓴다.

남성들은 부러진 깃대를 세워 전경들을 찔러 꿴다. 전경과 전경과 방패와

전경의 어머니와 유모차와 신생아가 한 꼬치를 이룬다. 전경의 어머니는

전경을 껴안으며 아들의 죽음을 애통해하고 사거리까지 달려갔던 신생아의

어머니는 되돌아와 아이를 부여잡고 대통령을 욕한다. 저놈들이 우리 아일

죽이려 한다. 저놈들이 우리 아일 죽이려 한다.



이때 저 유모차를 잡은 이들이 왜 하나같이 여자였는지 묻는 사람은 아무도

없었다



한 대의 유모차 안에 아이가 잠들어있다. 유모차 주위로 전경의 무리가

뒤덮고 있다. 너도 커서 군대에 가겠지. 전경은 꿀보직이니 넌 언젠가 내가

되겠지, 일제히 아이에게 낮은 목소리로 주문을 외웠다. 넌 언젠가 내가 될

것이다. 넌 언젠가 내가 될 것이다. 한 전경은 눈 앞에 한 무리의 유모차가

대열을 향해 오는 걸 보며 이렇게 중얼거렸다. 넌 언젠가 내가 될 것이다.

그리고 이어지는 비웃음. 호통. 정적. 넌 언젠가 내가 될 것이다.



한 전경이 유모차의 아이에게 수갑을 채운다. 아이의 손목에 맞게 제작된

수갑은 손목에 무리가 가지 않도록 통 알루미늄을 6-Axis CNC로 깎아

만들었다. 이 수갑은 한국 경공업의 위대한 산물이다. 뒤늦게 아이를 찾은

어머니가 아이의 손목에 채워진 수갑을 보며 오열한다. 우리 아이는 아토피가

있어요. 우리 아이는 금속 알러지가 있어요. 이미 어머니의 손목에는 수갑이

채워져 있다. 전리층에 도달한 거대한 스피커가 미란다 원칙을 고지한다.

아이의 손목에 채워진 수갑의 틈을 비집고 수포가 증식하기 시작한다. 이

아이는 금속 알레르기가 있어요. 몇 달 뒤 조달청 관계자는 어린이용 수갑에

알루미늄이 아닌 크롬이 사용된 것과 관련되어 그의 중학교 동창과 함께

나란히 법정에 서게 된다. 거대한 수포가 증식해 사거리를 가득 채운다. 군집

사이사이에서 수포들이 자라나고 전경들은 수포에 매몰된다. 종로의

사거리마다 거대한 덩어리들이 생겨났다는 소문이 들린다. 사진도, 발언도

없었지만 낙태된 아이의 이름처럼 결코 입 밖에 나오는 일은 없었다. 이런

이상한 기류를 포착한 사람들이 많아지고, 사람들은 각각 사거리들로

이동한다. 어머니들은 수포에 휘말린다. 아버지들은 수포에 휘말린다.

전리층으로부터 울려 퍼지는 거대한 목소리가 미란다 원칙을 고지한다.

이렇게 거대한 수포는 집어삼킨 사람들의 안면을 아스팔트에 눌러 긁으며

느릿느릿 한 방향으로 이동하기 시작한다. 우린 확성기의 구호에 귀를

기울이다 소리가 난 방향을 따라 걷는다. 우린 웃고 떠들다 확성기의 구호에

귀를 기울인다. 마치 대책위는 목소리로만 존재하는 것처럼 보였다. 이 시각

대책위는 목소리만 남아있었다.

