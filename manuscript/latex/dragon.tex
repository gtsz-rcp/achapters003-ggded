\hypertarget{uxbcbduxc7a5-uxc18duxc758-uxb4dcuxb798uxace4}{%

\section{벽장 속의

드래곤}\label{uxbcbduxc7a5-uxc18duxc758-uxb4dcuxb798uxace4}}



요 며칠간 이제 영원히 이에 대해 이야기할 수 없을 것이라는 생각에

이르렀다. 몇 달간 회사에 적응하느라 눈코 뜰 새 없이 바빴다. 겨우 퇴근 후

몇 시간을 낼 수 있을 만큼 여유를 찾은 후 나는 그 동안 미뤄두었던 글을

쓰고자 책상에 앉았다. 몇 편의 글들이 첫 문장에서 더 이어지지 않았고, 한

문장에서, 마지막 문단에서 끊어졌다. 아직도 벽장 속의 숨소리가 멈추질

않는다.



첫 번째 쓰고자 했던 글은 일종의 결산서였다. 그 동안 나의 드래곤를 돌보기

위해 지불했던 비용들을 결산해 나간 것이다. 식비 내역은 3년을 기점으로

해마다 기하급수에 근사하게 증가한다. 그가 박살낸 기물의 크기는 1년을

기점으로 산술급수에 근사하게 증가하고 결국 수천 만원의 빚이 남는다. 난

그 돈을 벌기 위해 때론 몸을 팔고, 신장을 팔 방법을 찾고 내놓을 수 있는,

내놓게 될 모든 것들에 값을 매기기 시작한다.



이 글은 결국 내가 빚더미에 앉고, 집 밖으로 내동댕이쳐지고, 나의 모든

모색이 실패로 돌아가는 것으로 마무리 될 것이었다. 이 결말 앞에서

우물쭈물하다 내가 이에 대해 온전히 서술할 만큼 끝장난 인생이 아니라는

비겁한 낙관을 얻었다. 글 속의 나는 탈출구는 있었을 테지만 그게 눈에 띌

만큼 오래 버티지 못했던 것이다. 글은 결국 마무리 되지 못한 채 지출

내역을 담은 표만 남아 있는 상태다.



아직도 벽장 문이 삐걱 인다. 난 다시 벽장으로 가 걸쇠를 점검하고 벽장을

묶은 강철 와이어를 다시 조였다.



두 번째로 난 일종의 픽션을 쓰기로 했다. 첫 번째 썼던 파국적인 이야기에서

좀 더 실질적인 낙관을 찾아내고자 했다. 아직도 항상 갖고 있는 생각이다.

'그 때 내가 그러지 않았다면 너와 나는 이 지경에 이르지 않았겠지.' 이런

이야기들의 연속인 것이다. '내가 너와 만나지 않았다면', '너와 눈이

마주치지 않았다면', '나의 이야기에 네 눈이 빛나지 않았다면', '그런 마법

같은 순간이 없었다면'. 그러지 않았다면 난 더 좋은 기회를 만났을 것이다.

하지만 너와 만난 이후 난 파국적인 결말의 근처에 이르렀고 결국 너를

원망할 수 밖에 없었다. 게다가 이런 엉뚱한 생각에 사로잡히지도 않았을

것이다.



어쩌면 나는 네 몸집에 걸맞게 더 큰 땅으로 보내야 했다고, 고집부리며 너를

보듬고 있는 게 아니었다고, 너를 큰 바다로 보내야 했다고, 파도 앞에서

엉엉 우는 너를 밀어 넣어야 했다고, 좀 더 일찍 그래야 했다고. 이런

이야기들이 이어지던 와중 나는 다시 한 번 너를 벽장 속으로 떠밀던 그 날을

떠올리게 되었고 그 동안 미뤄왔던 하나의 질문과 마주쳤다. 만일 이 모든

상황의 한복판에 있는 내가, 이 모든 것들을 기억하고 후회하는 내가 다시 그

상황과 마주친다면? 벽장 앞의 내가 벽장 앞의 너에게 벽장 속으로 들어가라

말하지 않을 수 있었을까?



난 두 번째 글을 쓰길 포기했다. 지금의 내가 지금의 벽장 앞에서 지금의

너와 마주친다 하더라도 나는 너에게 이 벽장 속으로 들어가라고, 기꺼이

내가 문을 열어 너를 벽장 속으로 밀어 넣었을 것이었다. 모두 무력한

몽상들이었다. 이 순간 내 손 마디마디를 갈갈이 찢어버리고 다신 아무

변명도 하지 않고 두 손으로 혀를 뽑아내고 싶다는 생각에 사로잡혔고 두

번째 글을 쓰기를 완전히 포기했다.



어쩌면 몇 년이 지나고 나면 나는 보다 더 태연하게 벽장을 점검할 수 있을

것이다. 나의 입은 그대로일 것이고, 나의 손도 그대로일 것이다. 난 어쩌면

이에 관해 이야기할 수 있을지도 모른다. 보다 더 세련된 문장으로 독자들을

설득할 수 있을지도 모른다. 그들은 나의 더러운 이야기를 듣고 눈물 흘리며

나를 위로할 것이다. 별 수 없었다고, 당신은 그럴 수 밖에 없었다고. 어쩌면

그보다 먼저 벽장 속의 그가 쇠약해질 것이다. 어쩌면 결국 그는 죽을

것이다. 그는 결국 죽게 될 것이다.



세 번째 썼던 글은 일종의 결산을 다시 쓴 것이었다. 그리고 네 번째는 두

번째 글을 다시 쓰고자 한 시도였다. 다섯 번째 글은 마지막 문단만 작성하고

중단한 글로 그를 벽장 속에 넣는 순간에 대해 서술했다. 다섯 번째 글의

아직 남아있는 문단은 아래와 같다.



그녀는 그를 벽장 앞으로 데려갔다. 그도 이미 알고 있는 눈치였다. 이런

일이 처음도 아니었고, 어느 날 갑자기 일어난 일도 아니라는 것쯤은 그도

알고 있었다. 언젠가 그가 말했다. 난 지금 너무 크고, 계속해서 더 커질

거야. 그 때도 넌 내 옆에 있을까?



그녀는 그에게 요구했다. 들어가. 그가 말했다. 여긴 너무 좁아. 그녀는

그에게 요구했다. 들어가. 그가 말했다. 여긴 너무 좁다고, 씨발년아. 둘은

모두 고개를 숙인 채 신경질 내며 소리질렀다. 그때 그들은 신경질이 나지

않았더라도 버럭버럭 크게 소리질러야만 했다. 그러지 않고는 그녀는 그를

벽장 속으로 밀어 넣을 수 없을 것 같았고, 그는 벽장 앞에서 버틸 수 없을

것 같았다. 그녀는 자신이 하고 있는 일이 무엇인지 잘 알고 있었다. 그녀는

아주 비겁하고, 아주 추악하고, 흉측하고, 사악한 일을 하고 있는 것이다.

그녀가 가장 사랑하는 존재에게 할 수 있는 가장 나쁜 짓을 하고 있는

것이다. 이런 다짐 없이는 절대 그를 벽장 안으로 밀어 넣을 수 없었다.



이 순간은 마치 그와 함께 한 삶과 비슷했다. 그녀는 아주 거대한 영향력

안에 있고 이미 일은 시작 된지 오래다. 일은 이미 손쓸 수 없는 상황에

이르렀고, 아무것도 할 수 없을 것처럼 보이는 상황 안에서 그럼에도 상황을

더 나아지게 하기 위해선 무엇이라도 해야 한다. 그리고 그 상황이 그녀에게

제시하는 몇 가지 선택들 중 하나를 골라 `이건 온전히 나의 선택'이라며

스스로의 가슴팍에 꽝꽝 때려 박는 것이다. 그녀가 이걸 가슴팍이 부서지도록

때려 박지 않으면, 틈날 때 마다 떨어지려 하는 그 `결정'이 떨어져나가지

않도록 꽝꽝 때려 박지 않으면 그녀는 아무것도 해나갈 수 없을 것 같았다.

이렇게 상황은 상황들이 되었고, 그런 무력감과 치열한 낙관이 그들의 초라한

역사였다. 그는 그녀가 깨닫기도 전에 수십 미터 크기로 자라났다. 그때

그녀가 할 수 있는 결정은 그럼에도 불구하고 그를 잘 길러야만 한다는

것이었고, 그가 할 수 있는 일은 그럼에도 불구하고 잘 자라는 것이었다.

원치 않는 자식이 태어났을 때 할 수 있는 일은 그를 잘 기르는 것이라고

믿었다. 그리고 그녀는 그날의 결정을 기억하며 그를 벽장 안으로 밀어

넣겠다는 선택을 한 것이다. 이것은 온전히 그녀의 결정인 것이다.



일련의 자기변명 이후 화자인 '나'는 그가 벽장 속으로 들어가는 장면을

묘사한다.



결국 그녀는 벽장 안으로 걸어 들어가기로 했다. 해질녘 시작한 논쟁이 끝난

때는 동틀 무렵이었다. 벽장 안으로 들어가기 위해 날개를 집어 삼켰다. 긴

꼬리를 집어 삼켰다. 그의 몸통 절반을 집어 삼켰다. 긴 목을 접었다. 그녀는

그 모습을 단 하나도 외면하지 않고 지켜봤다. 그녀가 그에게 할 수 있는

마지막 존중이었다. 그에게 있어 영원히 개씨발년으로 남아야겠다는

뜻에서였다. 작은 아이만한 크기가 된 그는 어두운 벽장 속으로 걸어

들어갔다. 벽장 앞은 망가진 그의 조각들로 피범벅이 되어 있었다. 그녀가

벽장 문을 닫기 전 어둠 속에서 빛나는 그의 눈과 잠시 마주쳤는데, 그

순간은 거실의 액자와 같이 영원히 그녀를 신경 쓰이게 만들 것이다. 난

이렇게 나의 거대한 용을 벽장 속에 가두는 데 성공했다.



언젠가 그가 말했다. 난 지금 너무 크고, 계속해서 더 커질 거야. 그 때도 넌

내 옆에 있을까? 그녀는 아직도 분명히 기억한다. 그때 그녀는 이렇게

말했다. 당연하지, 네가 이런 거 궁금해 하지도 않게 할거야. 그녀 가슴에

박힌 결정들이 펄럭이는 걸 보면 서늘한 바람이 부는 계절이 온 모양이다.



여기까지가 다섯 번째 글의 마지막 몇 문단이다. 난 다섯 번째 글 이후로

나의 드래곤에 대한 글쓰기를 완전히 그만두기로 했다. 오늘 아침 그의 기침

소리가 들렸고, 벽장 문의 경첩이 부서졌다. 난 나의 머리통만한 강철 경첩을

그곳에 달았다. 언제고 이 문을 고쳐 영원히 닫아 놓을 것이다. 문 앞에

철판을 대고, 좀 더 돈을 벌어 벽장이 있는 방 안에 수십 톤의 쇳물을

부어놓을 것이다. 이렇게 난 그를 완벽하게 망각할 것이다. 난 매일 그의

숨소리에 귀 기울일 것이고 그의 죽음을 기대할 것이다.

