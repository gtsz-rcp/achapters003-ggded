\documentclass[12pt, b6paper, openany]{memoir}
\usepackage[cm]{fullpage}
\usepackage[top=1.5cm, bottom=2.2cm, inner=2cm, outer=2.5cm]{geometry}
\usepackage{titlesec}

\usepackage{kotex}
\usepackage[breaklinks=true]{hyperref}
  \hypersetup{colorlinks,%
    citecolor=blue,%
    filecolor=blue,%
    linkcolor=blue,%
    urlcolor=blue}

\renewcommand{\baselinestretch}{1.3}
\setsecnumdepth{part}
\setlength{\beforechapskip}{0pt}

\titleformat{\chapter}{\filright}{}{0pt}{\normalfont\large\bfseries}
\titlespacing*{\chapter}{0pt}{0pt}{2\baselineskip}

\setlength{\parskip}{1em}

\newenvironment{lyric}{%
	\setlength{\parindent}{0pt}
}{}
\newenvironment{article}{}{}

% set tableofcontent
\maxtocdepth{chapter}

\author{에이미 앰플}
\title{고깃덩이}
\date{2018-06-01}

\begin{document}
	\frontmatter
		\begin{titlingpage}
		\maketitle
		\end{titlingpage}
		\tableofcontents

	\mainmatter
\begin{article}
\hypertarget{uxc11cuxbb38}{%

\chapter{서문}\label{uxc11cuxbb38}}



이 작업들은 내가 앞으로 하지 않을, 하지 않아야 할 작업들이 될 것이다. 그러면 좀 더 행복해질 것이다. 뭐든 행복한 쪽이 낫다.



에이미 앰플


\end{article}
\begin{lyric}
\hypertarget{uxd589uxc6b4uxc744-uxbe4cuxb2e4}{%

\chapter{행운을 빌다}\label{uxd589uxc6b4uxc744-uxbe4cuxb2e4}}



먼지 묻은 눈만 끔뻑거리다 새벽을 보낸다\\

어젠 바람을 베어다 널었다\\

벽 너머 아기 고양이를 외면하듯\\

나는 행운을 빌 뿐이다


\end{lyric}
\begin{lyric}
\hypertarget{uxb118uxc5b4uxc9c4-uxb0a0}{%

\chapter{넘어진 날}\label{uxb118uxc5b4uxc9c4-uxb0a0}}



하늘은 뿌옇고 별이 쏟아진 바닥엔 폭우의 흔적이 남았지 부서진 별빛이 낮은 음성으로 말했지 별도 하늘도 태양도 믿지 않는 우리는 굵은 밧줄을 메어 미래를 끌어내렸지 시커먼 그것을 그때 네가 그랬잖아: 우리가 일렬로 마주섰던 때 넌 웃으며 너는 내가 될 거라 그랬잖아 난 웃으며 나는 내가 될 거라 답했잖아 넌 웃으며 나는 네가 될 거라 말했잖아. 망가진 버스 정류장은 결국 하수종말처리장을 멈췄지 하수종말처리장만이 우리에게 응답했지 낮게 우는 별빛이 거기서 악취를 풍기며 빛났지. 우린 썩은 오니 위를 떠다니는 표지판이 기다란 팔로 인사하는 걸 보고 왔지



물론 우린 반갑게 답했어



안녕 안녕 안녕



우리가 목소릴 따라 걷던 새벽의 거리는 동이 틀 무렵 곧게 일어섰지 네가 일출을 따라 수직으로 선 도로를 우주에서 봤다면\\

넌 금빛 창문을 짚고 우릴 가리킬 수 있었을까 우린 누워 긴 대로의 끝을 걷는 한 무리의 소년 소녀를 보았어 이제 막 일년이 지났다



작년에 잘 저며 말려 둔 바람을 꺼내다 곱게 빻아 화분에 묻었어 씨앗도 흙도 없는 화분도 언젠간 응답할테지.


\end{lyric}
\begin{lyric}
\hypertarget{uxd5c8uxae30}{%

\chapter{허기}\label{uxd5c8uxae30}}



그렇게 우리의\\

허기진 시간이 반 토막 났다



우린 마지막으로 주린 뱃속을 헤집으며\\

지금 것 담아 놓은 이야기를 게워냈다



완전히 꺾인 목을\\

비집고 하얀 림프샘이\\

기어 나온다


\end{lyric}
\begin{lyric}
\hypertarget{sin-02}{%

\chapter{sin 02}\label{sin-02}}



우리의 시간이 단절을 알리며 내게 경고한다.



건너편 건널목은 이미 텅 비어 있다. 게슴츠레 내장을 보이는 보도블럭 쓰러진 빨간 가로등 청부폭력의 희생양 뿌리내린 깃발 화면에 금이 간 브라운관 바깥쪽으로 휘어진 가드레일 제자리로 돌아가는 횡단보도 내 뒤를 따라붙던 청소년 등 준비된 위기가 몸서리친다.



마지막 시간이 사인파를 그리며 내게 다가온다.



이때 허락된 유일한 운동이다


\end{lyric}
\begin{lyric}
\hypertarget{uxc774uxc6c3uxc758-uxc8fduxc74c}{%

\chapter{이웃의 죽음}\label{uxc774uxc6c3uxc758-uxc8fduxc74c}}



결국 오늘 밤 왼쪽 턱이 부풀어 올랐다 림프샘은 교외의 가로등처럼 하얗게 쏟아져 내렸다\\

몸살이 몰려올 내일\\

터진 림프샘에서 꿀 같은 연민이 흐를 것이다\\

여름 내 아귀처럼 날 탓하던 이웃이 등장할 시간이다\\

목을 빳빳이 세운 가로등이 타 들어간다 (어쨌든 서리 내린 꿀 위에 네온 사인이 빨갛게 달아오른다)\\

이웃은 중앙선 사이로\\

가지런히 걸어온다\\

타이레놀을 씹어 먹던 내 머리 위로 가로등이 쏟아져 내린다\\

엄지 발가락 위에 꽂힌 필라멘트는 아직도 시뻘겋게 타고 있다\\

뜯다 버린 발톱처럼 내팽개쳐진 가로등\\

조각을 들어 턱을 저민다\\

하얀 림프샘이 터지고\\

연민이 꿀같이 흐른다\\

이웃은 이미 죽고 난 후였다


\end{lyric}
\begin{lyric}
\hypertarget{sin-01}{%

\chapter{sin 01}\label{sin-01}}



우리의 지표면은 지금껏 맘껏 쌓인 지표면으로부터 높이 떨어져 있다.\\

매일 아침 투신자살을 기도하는 여자와\\

매일 저녁 껍질을 드러낸 여자의\\

시간이 교차한다\\

어느 날 새벽은 그녀가 내 발목을 붙잡고\\

이곳으로 기어오르는 걸 목격한 적이 있다



이곳의 대기는 중력마저 외면한다\\

어떤 것의 개입도 없이 교차점도 찍히지 않은 채\\

보는 그대로 불가능해왔다



지표면은 규칙적인 수직운동을 한다\\

이곳에 허락된 유일한 운동일 것이다



이따금 정지한 너를 본다\\

넌 지금껏 그 자리를 유지해왔다\\

나 또한 그래왔다\\

지표면은 규칙적인 수직운동을 한다\\

이곳에 허락된 유일한 운동일 것이다



그때의 시간을 기억한다\\

정지된 시간이 우리를 휘감았다\\

이후 우리의 정지된 시간은 이곳에 허락된 수직운동과 함께 해 왔다.\\

모두 나와는 전혀 상관 없는 것들이다



부디 오해하지 않길 바란다\\

단 한 번도 널 지나친 적이 없다 내게 허락된 유일한 해명일 것이다


\end{lyric}
\begin{lyric}
\hypertarget{section}{%

\chapter{72}\label{section}}



72는 종종 떠오르는 습관이 있다.



습관이란 게 어찌 해볼 만한 것이 아니다. 딱히 고칠 것도 없고, 위협할 만한 것도 아닌 실은 또 그런 것도 아닌. 목덜미의 꽁치뼈같은. 뭐 그쯤 존재하고 있을 것이다.



지축을 향한 수직선이라고 할 만한 형광등 줄이 있다. 우린 그 자리에 수건을 걸어 말린 다거나 가려운 정수리를 긁거나 줄을 올라타는 벌레를 태워 죽이거나 목을 매달다 포기하곤 한다.



가끔 땅이 크게 울릴 때가 있다. 아주 가끔이다. 72는 가끔 여자를 데려오고 희멀건 콘돔을 묶어 놓는다. 이 정도로 사소한 일이다.



72는 종종 지축을 향한 수직선을 따라 떠오른다. 이 또한 뭐라 말할 것 없이 사소한 일 중 하나다. 한 번은 72가 섹스를 하다 사정을 할 때쯤 누워있던 여자가 소리를 질렀다. 72가 떠올랐다. 정액을 흘리며 천장에 닿은 72는 마지막 한 방울이 가는 실을 타고 그녀의 배꼽에 닿을 때쯤 침대로 내려왔다.



몇시간 뒤 신호가 울렸다.\\

72는 메세지를 입력하고 있었다.\\

72는 메세지를 입력하고 있었다.\\

72의 접속이 끊어졌다.\\

며칠 뒤 달이 시청역 한가운데 쏟아졌다.



시청광장을 둘러싼 아스팔트와 아침의 잔디와 정비가 덜 된 지하철과 형광등과 멈춘 기중기와 리스기간이 남은 현대차와 폭리를 취하던 노점상과 깊숙이 머물던 조선의 유물들이 멈춘 쓰나미처럼 일어섰다.



검은 쓰나미의 가장자리에 맑은 유리가 맺혔다. 아마 나는 이번이 마지막 세대일 테지만 쓰나미 가장자리는 천천히 긴 궤적을 그리며 지면을 향할 것이다.


\end{lyric}
\begin{lyric}
\hypertarget{uxc885uxcc29uxc9c0}{%

\chapter{종착지}\label{uxc885uxcc29uxc9c0}}



경찰이 올려 보낸 수천 대의 드론이 미란다 원칙을 고지한다.



길게 늘어선 유모차의 행렬을 전경이 가로막았다. 깃발을 들고 따라 걷던 남성 몇 몇이 전경을 향해 소리질렀다. 유모차를 든 여성들은 일제히 허리를 숙여 아이들의 귀를 막았고, 유모차 가까이 딸랑이를 흔들어주었다. 한 남성이 소리쳤다. 왜 우리를 가로 막는가. 전경은 말이 없었다.



전경은 십 수명의 남성들로 구성된 군집생명체처럼 덩어리져 이 길에서 저 길로 일사 분란 하게 이동했다. 이들은 하나로 따로 떨어져 나와야만 비로소 내면의 인간성이 발동되는 듯 했지만 결국 이들을 대신해 입을 여는 건 이들의 어머니들이었다. 왜 우리 아들에게 손을 대느냐.



유모차를 미는 여성들이 아이들의 귀를 막고 소리쳤다. 왜 우릴 가로막는가. 전경들은 말이 없었다. 유모차의 줄이 전경의 대열 코 앞까지 밀고 들어왔고, 전경들의 방패가 잔뜩 긴장되었다.



전경들의 대열 뒤에서 거대한 마이크가 올라와 군중을 향해 침을 튀기며 말한다. 천이백이십일차 해산 통보입니다. 천이백이십일차 해산 통보. 여러분들은 시민들이 이용하는 찻길을 가로막고 있습니다.



뒤쪽에서 검은 팔뚝이 튀어나와 유모차를 휘어잡았다. 유모차를 잡은 여성들은 신경질적으로 주저앉아 소리를 지른다. 아이들이 비명을 지르며 경기한다. 입술이 파래진 아이가 바닥에 내동댕이쳐진다. 텅 빈 유모차들이 바닥에 나뒹군다. 한 무리를 이루었던 여성들은 모두 유모차를 집어 던지고 이제 한 아이의 어머니가 되어 전경의 군집을 벗어나기 위해 안간힘을 쓴다. 남성들은 부러진 깃대를 세워 전경들을 찔러 꿴다. 전경과 전경과 방패와 전경의 어머니와 유모차와 신생아가 한 꼬치를 이룬다. 전경의 어머니는 전경을 껴안으며 아들의 죽음을 애통해하고 사거리까지 달려갔던 신생아의 어머니는 되돌아와 아이를 부여잡고 대통령을 욕한다. 저놈들이 우리 아일 죽이려 한다. 저놈들이 우리 아일 죽이려 한다.



이때 저 유모차를 잡은 이들이 왜 하나같이 여자였는지 묻는 사람은 아무도 없었다



한 대의 유모차 안에 아이가 잠들어있다. 유모차 주위로 전경의 무리가 뒤덮고 있다. 너도 커서 군대에 가겠지. 전경은 꿀보직이니 넌 언젠가 내가 되겠지, 일제히 아이에게 낮은 목소리로 주문을 외웠다. 넌 언젠가 내가 될 것이다. 넌 언젠가 내가 될 것이다. 한 전경은 눈 앞에 한 무리의 유모차가 대열을 향해 오는 걸 보며 이렇게 중얼거렸다. 넌 언젠가 내가 될 것이다. 그리고 이어지는 비웃음. 호통. 정적. 넌 언젠가 내가 될 것이다.



한 전경이 유모차의 아이에게 수갑을 채운다. 아이의 손목에 맞게 제작된 수갑은 손목에 무리가 가지 않도록 통 알루미늄을 6-Axis CNC로 깎아 만들었다. 이 수갑은 한국 경공업의 위대한 산물이다. 뒤늦게 아이를 찾은 어머니가 아이의 손목에 채워진 수갑을 보며 오열한다. 우리 아이는 아토피가 있어요. 우리 아이는 금속 알러지가 있어요. 이미 어머니의 손목에는 수갑이 채워져 있다. 전리층에 도달한 거대한 스피커가 미란다 원칙을 고지한다. 아이의 손목에 채워진 수갑의 틈을 비집고 수포가 증식하기 시작한다. 이 아이는 금속 알레르기가 있어요. 몇 달 뒤 조달청 관계자는 어린이용 수갑에 알루미늄이 아닌 크롬이 사용된 것과 관련되어 그의 중학교 동창과 함께 나란히 법정에 서게 된다. 거대한 수포가 증식해 사거리를 가득 채운다. 군집 사이사이에서 수포들이 자라나고 전경들은 수포에 매몰된다. 종로의 사거리마다 거대한 덩어리들이 생겨났다는 소문이 들린다. 사진도, 발언도 없었지만 낙태된 아이의 이름처럼 결코 입 밖에 나오는 일은 없었다. 이런 이상한 기류를 포착한 사람들이 많아지고, 사람들은 각각 사거리들로 이동한다. 어머니들은 수포에 휘말린다. 아버지들은 수포에 휘말린다. 전리층으로부터 울려 퍼지는 거대한 목소리가 미란다 원칙을 고지한다. 이렇게 거대한 수포는 집어삼킨 사람들의 안면을 아스팔트에 눌러 긁으며 느릿느릿 한 방향으로 이동하기 시작한다. 우린 확성기의 구호에 귀를 기울이다 소리가 난 방향을 따라 걷는다. 우린 웃고 떠들다 확성기의 구호에 귀를 기울인다. 마치 대책위는 목소리로만 존재하는 것처럼 보였다. 이 시각 대책위는 목소리만 남아있었다.


\end{lyric}
\begin{article}
\hypertarget{uxd6c4uxc77cuxb2f4}{%

\chapter{후일담}\label{uxd6c4uxc77cuxb2f4}}



우리는 아주 오랫동안 키스했다. 그가 나의 등을 쓸어 내렸다. 나는 그의 옆구리를 쓸어 내렸다. 나는 그의 머리를 쓰다듬었다. 그의 팔꿈치를 간질였다. 그의 어깨가 움직이는 것이 느껴진다. 그가 숨을 들이쉬자 그의 배가 나의 배에 닿는다. 머리 속이 온통 하얗다. 나는 그를 자빠뜨렸다. 아랫배에서 커다란 나무가 자라는 것 같다. 나는 입술에 침을 잔뜩 묻혀 그의 이마에, 눈꺼풀에 코끝에, 양 볼에, 턱 끝에 입맞추었다. 우린 아주 오랫동안 키스했다. 잠시 우리가 입술을 떼었을 때 가느다란 침이 우리 입술 사이로 늘어졌다. 늘어진 침이 끊어지는 소리가 방 안에 울렸다. 내가 그를 응시했고, 그가 나를 응시했다. 그 때 우리 사이엔 침과 얇은 점막 말고는 끼어들 것이 아무것도 없었다. 입술이 차가워졌다. 이대로 식을 수는 없었다.



그가 내게 말했다. 나는 그에게 말했다. 나는 그가 더 말하길 원했다. 그의 탄식에는 부피가 있다. 그의 목소리는 내 온 몸을 울린다. 방 안이 안개로 가득 찬다. 귓가의 배게 소리가 너무 크다. 한 여름 녹아 내린 솜처럼 그의 목소리가 내 귓 속을 틀어막았다. 내가 그의 엉덩이를 쥐자 그가 다리를 길게 뻗었다. 그의 무릎이 내 정강이에 닿는다. 나는 온 힘을 다해 그를 끌어안았다. 그가 나를 밀고 올라온다. 그의 침이 식어 귀가 싸늘하다.



\hypertarget{section}{%

\section{}\label{section}}



너는 내 다리를 벌린다. 너는 내 어깨를 어루만진다. 너는 내 배를 쓸어준다. 너는 내 가슴을 쥔다. 단단해진 내 유두가 네 등을 할퀸다. 너는 내 보지에 손을 넣는다. 조심스러워야만 한다. 나는 조용히 한숨을 쉰다. 그것이 너를 자극할 것이다. 너는 나를 핥는다. 너는 내 머리카락을 만진다. 너는 종종 나와 눈을 마주친다. 너는 잊지 않게 내게 입술을 맞춘다. 너는 내가 얼마나 아름다운지 말해주어야만 한다. 너는 내 발가락을 주의 깊이 매만져야 한다. 너는 살짝 부푼 내 아랫배를 어루만져야만 한다. 너는 내 까칠한 음모를 쓸어 내린다. 매번 너는 놀라워해야 한다. 하지만 익숙하게 나를 대해야 할 것이다.



나는 네 것을 쥐고 핥는다. 네 음낭을 입에 넣었다 뺀다. 이미 네 보지는 짭쪼름하게 젖어 있다. 나는 너에게 그것에 대해 말한다. 너는 필히 부끄러워해야만 한다. 내가 네 것을 빠는 동안 너는 내 머리를 어루만진다. 너는 내 어깨를 쓰다듬고 너는 내 귓불을 만진다. 나는 간혹 부끄러운 표정으로 너를 올려볼 것이지만 너는 그런 나를 받아주어야만 한다.



나는 그에게 명령했다. 모두 흘러내리기 전에 빨아서 내게 줘, 어서. 그는 그렇게 하기로 하였다. 그의 것이 빠져나간 자리 잔뜩 젖은 혀가 기어들어왔다. 더욱 깊이 들어올 수 있도록 그의 머리를 힘껏 끌어당겼다. 실은 그 때 그가 내 다리 사이에서 질식해 죽더라도 상관 없다고 생각했다. 그가 내 허벅지를 세게 쥐었다. 나는 덜컥 겁이 나 그를 밀어내었다. 얼굴이 빨갛게 된 그가 거친 표정으로 나를 보았다. 그의 끈적한 침이 세어 내 항문을 간질였다. 내가 누운 그의 위에서 허리를 움직일 때도 그랬다. 긴 머리가 등을 간질였다. 나는 내 머리를 틀어 잡았다. 온 몸을 적신 땀이 내 등골을 타고 엉덩이 사이로 흘러 들어왔다. 그의 가슴팍에 붉은 꽃이 피어 있었다. 내가 허리를 세게 움직일수록 그의 음낭은 나의 항문과 더욱 가까워졌다. 내 것은 완전히 젖어 있었고, 그의 음모는 그의 아랫배와 허벅지에 달라붙어 있었다. 우리는 서로 얽혀 떨어지지 않았다. 그의 배를 적신 땀이 배꼽에 고였다. 나의 등골을 따라 흐르는 땀은, 그를 흠뻑 적신 나의 액체는 그의 항문을 간질였을 것이다. 내게 그러했듯 말이다.



\hypertarget{section-1}{%

\section{}\label{section-1}}



우린 가죽만 남은 짐승들이었다. 우린 여름의 호숫가였다. 우린 죽은 신을 위한 찬송가였다. 우린 젖은 이불이었다. 우린 썩은 고깃덩어리였다. 우린 불꺼진 거리였다. 우린 구겨진 화장지였다. 우린 며칠 째 복용을 미룬 피임약이었다. 우린 진득한 점액이었다. 우린 새까매진 무릎이었다. 우린 멍든 허벅지였다. 우린 짓무른 엉덩이였다. 우린 갈라진 입술이었다. 우린 마른 침이었다. 우리는 상처투성이 뒤꿈치였다. 우린 망가진 란제리였다. 튀어나온 와이어가 턱을 할킨다. 우린 낮은 탄식이었다. 우린 울상이 된 얼굴이었다. 우린 젖은 이불이었다. 우린 더러운 변기였다. 우린 탁한 목욕물이었다. 우린 아침의 단내였다. 우린 시큼한 겨드랑이였다. 우린 먼지 묻은 맨발이었다. 우린 턱까지 흐른 침이었다. 우리는 잔뜩 긴장한 팔이었다. 우리는 풀어헤쳐진 머리카락이었다.



우리는 죽은 아이를 요리하는 어머니였다. 나는 그를 말단부터 씹어 삼켰다. 우리는 아주 공들여 우리를 잡아먹었다. 다시 있어선 안 될 개인사였다. 나는 그에게 명령했다. 모두 흘러내리기 전에 빨아서 내게 줘, 어서. 그는 그렇게 하기로 하였다. 그는 정성껏 내 것을 빨았다. 입에 담아 내게 키스했다. 우린 아주 오랫동안 그것을 나누었다. 비릿한 냄새가 진동했다. 혀 끝이 아려왔다. 입 안이 텁텁했다. 목구멍이 간지러웠다. 그 또한 그랬을 것이다. 내 것이 차갑게 식어가고 있다. 그 또한 그럴 것이다. 이대로 식을 수는 없다.


\end{article}
\begin{lyric}
\hypertarget{uxc0acuxac74uxc758-uxc9c0uxd3c9uxc120}{%

\chapter{사건의 지평선}\label{uxc0acuxac74uxc758-uxc9c0uxd3c9uxc120}}



너의 입술이 건조해지기 시작한 때를 눈치채지 못한 채 떨어진 입술을 다시 주워 오려 단단한 겨울의 강을 파헤쳤다. 우린 두 팔이 서로를 감은채로 영원한 겨울을 불러들였다 너의 침묵은 올해의 첫눈처럼 바닥에 닿기도 전에 망가졌다. 우린 썩은 달걀처럼 악취미 가득한 선물이 될 것이다 우린 사랑을 가장한 채로 들키지 않은 채로 영원히 살게 될 것이다



차라리: 우리가 기어이 이 말을 택하기 까지 바친 긴 불면의 새벽들, 차라리



애써 널 돌아 세워 나를 다시 보라 말하지 못한 어제는 차라리. 시절의 죽음 앞에 마주쳤던 내가 세 번 비겁해질 수 있었다면 어쩌면 난 이 자리에 머물지 않았을 텐데 너 또한



우린 사건의 지평선에서 만날 수 없는 소식의 제목을 말한다. 우리의 목소린 서로의 어깨를 스치며 무한히 늘어난 목을 교차해 묶는다.



우린 사건의 지평선에서 오래된 탄생의 비명과 마주친다. 비명은 우리의 머리카락을 흩어 놓고 무한히 연기된 죽음을 앞에 놓는다



우린 목 메인 웃음으로 축복의 말을 건네며 저주는 내려놓고 돌아서겠지 우린 이어진 불면의 끝에 배교자의 죄책감으로 숙면을 포기하겠지



: 차라리



우리가 기어이 이 말을 택하기 까지 버린 수많은 숙고의 순간들. 차라리 애써 널 돌아 세워 날 다시 보라 말하지 못한 어제는. 차라리 세계의 죽음 앞에 마주쳤던 내가 세 번 비겁해질 수 있었다면, 어쩌면, 난 이 자리에 머물지 않았을 텐데. 너 또한


\end{lyric}
\begin{article}
\hypertarget{uxce90uxb098uxb2e4uxad6cuxc2a4uxb2e4uxc6b4}{%

\chapter{캐나다구스다운}\label{uxce90uxb098uxb2e4uxad6cuxc2a4uxb2e4uxc6b4}}



벌써 작년 일이네. 작년 겨울에 엄청 추웠었잖아. 팔다리 막 꽁꽁 얼고 그랬었잖아. 작년 이맘때 쯤 여자친구랑 겨울 한복판에 서 있었거든. 걔를 보니까 엄청 이쁜거야. 완전 이뻐 그래서 뽀뽀를 했단 말이야. 그날 얼마나 추웠는지 걔 입술이 땅에 톡 떨어져버린거야. 앞니에다 뽀뽀할 순 없잖아. 그 느낌 좀 곤란하잖아. 걔도 놀래고 나도 놀래서 막 바닥을 벅벅 기면서 입술을 찾아다녔어. 근데 그게 어디로 갔는지 도통 보이질 않는거지. 그 동안 걔는 막 악악악 뭔가 말을 하는데 처음엔 입술 좀 찾아달라고 그러는 줄 알았거든? 근데 그게 아냐. 애오야 알 알이 이어! 이러는데, 아오 답답해서 내 입술을 떼다가 걔한테 줬어. 내 것도 꽁꽁 얼어서 잘 떨어졌단 말이지. 내 입술을 달고 걔가 한다는 말이 채소야, 나 할 말이 있어. 우리 헤어지자.



아니, 왜? 대체 내가 뭐가 문젠데? 문제가 있으면 말을 하고, 서로서로 그렇게 잘 고쳐나가면서 사는게 우리네 이쁜 인생이잖아? 우리 그렇게 이쁘게 살고 싶은거잖아? 나도 할 말이 많아져서, 그 입술 좀 달라고, 내 말 좀 들어달라고 소리쳤는데 걔는 그냥 나한테 헤어지잔 말만 하고 돌아서 가버렸어.



그때 엄청 추웠거든. 입술도 뚝뚝 떨어지고, 손끝도 바스러지고, 머리카락도 부서지고. 멍청하게 가만 서 있다 보니까 바람이 쌩쌩 불고 온 몸이 박살나버렸어. 산산조각나서 거기 그냥 내동댕이쳐진거야. 길바닥에 내가 툭툭 떨어져나가는 소리가 나는데도 걔는 뒤도 안돌아보고 가더라. 걔가 뒤를 돌아본다 한들 나나 걔나 할 수 있는 일이 아무것도 없었던 것은 분명하지만, 그래도 그렇잖아. 아냐 이 얘긴 관두자.



암튼 그 추운 겨울의 한복판에서 나는 바보같이 할 말이 있는데도 하지 못했다고, 그냥 두 눈만 꿈뻑꿈뻑 하다가 눈꺼풀도 박살나 날아다니고. 가만 생각해보니 내가 하려던 그 말을 한다고 뒤집어진 내 속이 가라앉을 것 같지도 않고, 게다가 우리가 연애한거잖아? 너네가 연애한게 아니라, 걔 혼자 연애한 것도 아니잖아. 우리가 헤어진거잖아.



나중에 봄이 오고 나서야 이것저것 산산조각난 것들이 서서히 녹아서 이래저래 흩어진 것들을 주워 모았지. 그런데 입술은 없었어. 그 썅년이 내 입술을 달고 튄거야. 아냐, 걔 썅년 아냐. 좋은 여자였어. 아주 훌륭한 여성이었지. 아주 훌륭해서 내 입술을 달고 튀었거든. 원망 같은거 없어. 농담이야. 진짠데? 농담인데? 웃으라고 한 말인데? 웃으면 되는데? 그렇게 깐죽대지 않아도 되는데? 암튼 그래서 입술이랑 발가락 몇 개는 근처에 이마트에서 사다가 붙였지. 며칠 달고 있으니까 걍 내 것같이 잘 움직이더라. 다행인 일이지.



아, 근데 걔는 왜 박살나지 않고 잘 돌아갔느냐고? 나만 박살나고? 걔는 구스다운 점퍼를 입었고, 나는 시발 존나 싸구려 솜점퍼를 입었거든. 그러니까 이번 겨울에는 구스다운 점퍼를 사라. 존나 따뜻하다. 옷 껴입고 내복 껴입고 그럴 필요 없어 걍 오리털 하나면 되는겨. 한 벌만 사지 말고 두 벌 사라. 존내 따뜻해.


\end{article}
	\appendix
	\backmatter
\end{document}
